% SPH R Appreciation Society "Packages of the month"
% (c) 2025 Malcolm Gillies <malcolm.gillies@unsw.edu.au>
% https://github.com/mbg-unsw/sphras
%
% This work is licensed under a
% Creative Commons Attribution-NonCommercial-ShareAlike 4.0
% International Licence
\documentclass[aspectratio=169,12pt]{beamer} % XXXX fix AR here
\usepackage{pgfpages}
\usepackage{fancyvrb}
\usepackage{pgfplots}
%\usepackage[latin1]{inputenc}
%\usepackage[T1]{fontenc}
%\usepackage{textcomp}
%\usefonttheme{serif} % need this with Charter font
\usetheme{auriga}  % using default now
\usecolortheme{auriga}  % using default now
%\usepackage[libertine]{libertine} % not using osf (old-style figures)
%\usepackage[scale=0.9]{tgheros} % scale to match libertine
%\usepackage[varqu,varl]{inconsolata}
%\usepackage[libertine]{newtxmath}
\usepackage{graphicx}
\usepackage{tikz}
\usetikzlibrary{shadows}
\usepackage{tikzpagenodes}
\usepackage{natbib}
\usepackage{gitinfo2}
\usepackage{xurl} % word wrap URLs
\usepackage{listings}

\lstset{ 
  language=R,                     % the language of the code
  basicstyle=\scriptsize\ttfamily, % the size of the fonts that are used for the code
  %numbers=left,                   % where to put the line-numbers
  %numberstyle=\tiny\color{blue},  % the style that is used for the line-numbers
  %stepnumber=1,                   % the step between two line-numbers. If it is 1, each line
                                  % will be numbered
  %numbersep=5pt,                  % how far the line-numbers are from the code
  backgroundcolor=\color{white},  % choose the background color. You must add \usepackage{color}
  showspaces=false,               % show spaces adding particular underscores
  showstringspaces=false,         % underline spaces within strings
  showtabs=false,                 % show tabs within strings adding particular underscores
  %frame=single,                   % adds a frame around the code
  rulecolor=\color{black},        % if not set, the frame-color may be changed on line-breaks within not-black text (e.g. commens (green here))
  tabsize=2,                      % sets default tabsize to 2 spaces
  captionpos=b,                   % sets the caption-position to bottom
  breaklines=true,                % sets automatic line breaking
  breakatwhitespace=false,        % sets if automatic breaks should only happen at whitespace
  keywordstyle=\color{blue},      % keyword style
  commentstyle=\color{green},   % comment style
  stringstyle=\color{green}      % string literal style
} 

\pgfplotsset{compat=1.17}
\hypersetup{pdfencoding=auto}

\renewcommand{\gitMark}{\color{lightgray}\texttt{\tiny\gitBranch\,@\,\gitAbbrevHash\,\gitAuthorDate}}

%\renewcommand{\bibsection}{} % suppress "References" section

%\setbeamertemplate{navigation symbols}{} % remove navigation symbols
%\setbeamercolor*{item}{fg=darkred}

\title{SPH R Appreciation Society:\\Small cell suppression}
\author{Malcolm Gillies <malcolm.gillies@unsw.edu.au>}
\institute{25 February 2025}

\newif\ifsidebartheme
\sidebarthemefalse

\newdimen\contentheight
\newdimen\contentwidth
\newdimen\contentleft
\newdimen\contentbottom
\makeatletter
\newcommand*{\calculatespace}{%
    \contentheight=\paperheight%
    \ifx\beamer@frametitle\@empty%
        \setbox\@tempboxa=\box\voidb@x%
      \else%
        \setbox\@tempboxa=\vbox{%
          \vbox{}%
          {\parskip0pt\usebeamertemplate***{frametitle}}%
        }%
        \ifsidebartheme%
          \advance\contentheight by-1em%
        \fi%
      \fi%
    \advance\contentheight by-\ht\@tempboxa%
    \advance\contentheight by-\dp\@tempboxa%
    \advance\contentheight by-\beamer@frametopskip%
    \ifbeamer@plainframe%
    \contentbottom=0pt%
    \else%
    \advance\contentheight by-\headheight%
    \advance\contentheight by\headdp%
    \advance\contentheight by-\footheight%
    \advance\contentheight by4pt%
    \contentbottom=\footheight%
    \advance\contentbottom by-4pt%
    \fi%
    \contentwidth=\paperwidth%
    \ifbeamer@plainframe%
    \contentleft=0pt%
    \else%
    \advance\contentwidth by-\beamer@rightsidebar%
    \advance\contentwidth by-\beamer@leftsidebar\relax%
    \contentleft=\beamer@leftsidebar%
    \fi%
}
\makeatother

\begin{document}

\setbeamercolor{page number in head/foot}{fg=white}

\begin{frame}[plain]
\titlepage
\end{frame}

\begin{frame}{Small cell suppression}
	\begin{itemize}
		\item What even is it?
		\item What packages are available
		\item Quick \texttt{easySdcTable} walkthrough
		\item Work in progress
	\end{itemize}
\end{frame}

\begin{frame}{Privacy, confidentiality, disclosure, reidentification}
	\begin{itemize}
		\item Privacy act requirements for `de-identification'
		\item Harm due to disclosure of personal information
		\item Data custodian requirements
		\item AIHW/ABS requirements
	\end{itemize}
\end{frame}

\begin{frame}{Statistical disclosure control}
	\begin{itemize}
		\item Deidentification, confidentialisation, ...
		\item Methods to achieve this aim include:
		\item \emph{Small cell suppression}
	\end{itemize}
\end{frame}

\begin{frame}{Rule of 6 (or 5, or 10, ...)}
	\center
	\begin{tabular}{lr}
	Age cat & Count \\
	\hline
	15-19 & 16 \\
	20-24 & 8 \\
	25-29 & \colorbox{orange}{3} \\
	30-34 & \colorbox{orange}{4} \\
	Total & 31 \\
	\hline
	\end{tabular}
\end{frame}

\begin{frame}{Secondary suppression}
	\center
	\begin{tabular}{lrrr}
	& Male & Female & Total \\
	Age cat & Count & Count & Count\\
	\hline
	15-19 & 16 & 0 & 16 \\
	20-24 & 8  & 10 & 18\\
	25-29 & \colorbox{orange}{3}  & 8 & 11 \\
	30-34 & \colorbox{orange}{4} & 5 & 9 \\
	Total & 31  & 23 & 54 \\
	\hline
	\end{tabular}
\end{frame}

\begin{frame}{There's a package (or two) for that!}
	\begin{itemize}
		\item GaussSuppression
		\item sdcTable
		\item easySdcTable
		\item pTable
		\item ACRO-R
		\item cellKey
		\item modulartabler
	\end{itemize}
\end{frame}

\begin{frame}[fragile]\frametitle{easySdcTable (1)}
\begin{lstlisting}[language=R]
> agesex
# A tibble: 8 × 3
  agecat sex    count
  <chr>  <chr>  <dbl>
1 15-19  Male      16
2 20-24  Male       8
3 25-29  Male       3
4 30-34  Male       4
5 15-19  Female     0
6 20-24  Female    10
7 25-29  Female     8
8 30-34  Female     5
\end{lstlisting}
\end{frame}

\begin{frame}[fragile]\frametitle{easySdcTable (2)}
\begin{lstlisting}[language=R]
> pivot_wider(agesex, id_cols=agecat, names_from=sex,
              values_from=count)
# A tibble: 4 × 3
  agecat  Male Female
  <chr>  <dbl>  <dbl>
1 15-19     16      0
2 20-24      8     10
3 25-29      3      8
4 30-34      4      5
\end{lstlisting}
\end{frame}

\begin{frame}[fragile]\frametitle{easySdcTable (3)}
\begin{lstlisting}[language=R]
> agesex.p <- ProtectTable(agesex, dimVar=c("agecat", "sex"),
              freqVar=c("count"), maxN=5, protectZeros=FALSE)
> pivot_wider(agesex.p$data, id_cols=agecat, names_from=sex,
              values_from=suppressed)
# A tibble: 5 × 4
  agecat Total Female  Male
  <chr>  <dbl>  <dbl> <dbl>
1 Total     54     23    31
2 15-19     16      0    16
3 20-24     18     10     8
4 25-29     11     NA    NA
5 30-34      9     NA    NA
\end{lstlisting}
\end{frame}

\begin{frame}{Next steps}
	\begin{itemize}
		\item easySdcTable + gtSummary = Table 1 !?!?
	\end{itemize}
\end{frame}

\end{document}
