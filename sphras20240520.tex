% SPH R Appreciation Society "Packages of the month"
% (c) 2024 Malcolm Gillies <malcolm.gillies@unsw.edu.au>
% https://github.com/mbg-unsw/sphras
%
% This work is licensed under a
% Creative Commons Attribution-NonCommercial-ShareAlike 4.0
% International Licence
\documentclass[aspectratio=169,12pt]{beamer} % XXXX fix AR here
\usepackage{pgfpages}
\usepackage{fancyvrb}
\usepackage{pgfplots}
%\usepackage[latin1]{inputenc}
%\usepackage[T1]{fontenc}
%\usepackage{textcomp}
%\usefonttheme{serif} % need this with Charter font
\usetheme{auriga}  % using default now
\usecolortheme{auriga}  % using default now
%\usepackage[libertine]{libertine} % not using osf (old-style figures)
%\usepackage[scale=0.9]{tgheros} % scale to match libertine
%\usepackage[varqu,varl]{inconsolata}
%\usepackage[libertine]{newtxmath}
\usepackage{graphicx}
\usepackage{tikz}
\usetikzlibrary{shadows}
\usepackage{tikzpagenodes}
\usepackage{natbib}
\usepackage{gitinfo2}
\usepackage{xurl} % word wrap URLs

\pgfplotsset{compat=1.17}
\hypersetup{pdfencoding=auto}

\renewcommand{\gitMark}{\color{lightgray}\texttt{\tiny\gitBranch\,@\,\gitAbbrevHash\,\gitAuthorDate}}

%\renewcommand{\bibsection}{} % suppress "References" section

%\setbeamertemplate{navigation symbols}{} % remove navigation symbols
%\setbeamercolor*{item}{fg=darkred}

\title{SPH R Appreciation Society:\\Powerpoint graphics in R}
\author{Malcolm Gillies <malcolm.gillies@unsw.edu.au>}
\institute{20 May 2024}

\newif\ifsidebartheme
\sidebarthemefalse

\newdimen\contentheight
\newdimen\contentwidth
\newdimen\contentleft
\newdimen\contentbottom
\makeatletter
\newcommand*{\calculatespace}{%
    \contentheight=\paperheight%
    \ifx\beamer@frametitle\@empty%
        \setbox\@tempboxa=\box\voidb@x%
      \else%
        \setbox\@tempboxa=\vbox{%
          \vbox{}%
          {\parskip0pt\usebeamertemplate***{frametitle}}%
        }%
        \ifsidebartheme%
          \advance\contentheight by-1em%
        \fi%
      \fi%
    \advance\contentheight by-\ht\@tempboxa%
    \advance\contentheight by-\dp\@tempboxa%
    \advance\contentheight by-\beamer@frametopskip%
    \ifbeamer@plainframe%
    \contentbottom=0pt%
    \else%
    \advance\contentheight by-\headheight%
    \advance\contentheight by\headdp%
    \advance\contentheight by-\footheight%
    \advance\contentheight by4pt%
    \contentbottom=\footheight%
    \advance\contentbottom by-4pt%
    \fi%
    \contentwidth=\paperwidth%
    \ifbeamer@plainframe%
    \contentleft=0pt%
    \else%
    \advance\contentwidth by-\beamer@rightsidebar%
    \advance\contentwidth by-\beamer@leftsidebar\relax%
    \contentleft=\beamer@leftsidebar%
    \fi%
}
\makeatother

\begin{document}

\setbeamercolor{page number in head/foot}{fg=white}

\begin{frame}[plain]
\titlepage
\end{frame}

\begin{frame}{officer/rvg for output to PowerPoint slides}
	\begin{itemize}
		\item What is it?
		\item What's great?
		\item What sucks?
		\item How does it work?
	\end{itemize}
\end{frame}

\begin{frame}{But first a digression about vector graphics\dots}
	\begin{itemize}
		\item Raster graphics (PNG, GIF, JPG) have a fixed resolution
		\item \emph{Changing the size or medium loses quality}
		\item Vector graphics are based on geometrical shapes, not pixels
		\item \emph{Great for sharp graphs and diagrams, not for images}
		\item \emph{You can edit vector graphics with Inkscape (free)}
	\end{itemize}
\end{frame}

\begin{frame}{Which vector graphics file format?}
	\center
	\begin{tabular}{|l|l|}
	\hline
	SVG & A standard, and quite a good one! \\
	\hline
	PDF & Widely supported and functional \\
	\hline
	EMF & Microsoft's own standard, but terrible \\
	\hline
	\end{tabular}
\end{frame}

\begin{frame}{Why not PDF?}
	\begin{itemize}
		\item No support for in-line PDF graphics in MS Word
		\item Some journals don't accept it for figures etc
		\item Some coauthors don't know how to edit it
		\item \emph{But I suggest you use PDF if you can}
	\end{itemize}
\end{frame}

\begin{frame}{Enter RVG\dots}
	\begin{itemize}
		\item Microsoft's new vector graphics format
		\item Good quality graphics
		\item \emph{Not widely known or supported by third parties}
		\item \emph{No stand-alone file format}
		\item \emph{Implementation in MS Word is half-baked}
	\end{itemize}
\end{frame}

\begin{frame}{Solution: Powerpoint + RVG}
	\begin{itemize}
		\item R graphical output $\rightarrow$ PPTX
		\item Easy to edit and annotate
		\item Can cut and paste into Word
	\end{itemize}
\end{frame}

\begin{frame}{officer/rvg: What is it?}
	\begin{itemize}
		\item \texttt{officer}, a package for writing MS Office files
		\item \texttt{rvg}, additional support for output to RVG
	\end{itemize}
\end{frame}

\begin{frame}{officer/rvg: What's great?}
	\begin{itemize}
		\item Simple and logical (more or less)
		\item Just a few lines of extra code
		\item Works with \texttt{officdown} package
	\end{itemize}
\end{frame}

\begin{frame}{officer/rvg: What sucks}
	\begin{itemize}
		\item Font handling can be tricky
		\item No graphical groups e.g. all points in a scatterplot
		\item Still one or two bugs e.g. text alignment
	\end{itemize}
\end{frame}

\begin{frame}{officer/rvg: Intros and tutorials}
	\begin{itemize}
		\item \scriptsize{\url{https://www.pipinghotdata.com/posts/2020-09-22-exporting-editable-ggplot-graphics-to-powerpoint-with-officer-and-purrr/}}
		\item \scriptsize{\url{https://www.apreshill.com/blog/2021-07-officedown/}}
	\end{itemize}
\end{frame}

\end{document}
